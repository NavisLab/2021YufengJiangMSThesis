\chapter{绪论}

\section{课题研究的背景与意义}
我们处于一个信息化时代,每个人手中的电子产品每天都会产生大量的视频、图像和音频等数据,如何利用这些数据是一个重要的研究课题。随着计算机硬件性能的迅速发展和计算能力的提高,基于深度学习的计算机视觉研究如火如荼地开展。计算机视觉是一门研究让如何机器“看”的科学,其研究工作在我们的生活中无处不在,如手机的指纹和人脸解锁、火车站的人脸识别等。在计算机视觉领域,很多问题如超分辨率、图像着色、风格迁移和图像修复等都可以视为输入一张图像,然后将其“转译”成相应的输出图像,即从一个域转译到另一个域,如同一句话可以用汉语、英语等语言表达,一个场景也可以用RGB图、梯度场图、语义标签图等渲染。与语言翻译类似,图像转译定义为在给定足够训练数据的情况下,将场景的一种可能的表示转译成另一种表示的任务。

在早期研究中,卷积神经网络是最常用于各种预测任务的网络结构,虽然卷积神经网络可以通过学习最小化损失函数得到预测结果,且学习的过程是自动的,但仍需要人为设计有效的损失,换句话说,损失函数是特定于某一个任务的,无法做到各种任务通用。如果只是简单的最小化预测结果与真实像素之间的欧式距离,很容易产生模糊的结果,因为欧式距离是通过平均所有可能的输出进行最小化的,为了更好的解决这个问题,我们可能需要一个通用的损失函数。2014年提出的生成对抗网络\cite{goodfellow2014generative}从考虑预测结果与真实数据是否可区分出发,设计了一个损失函数来满足此目标,因为生成对抗网络学习的是一个适应数据的损失函数,所以它可以应用于多数任务。2017年pix2pix\cite{isola2017image}模型首先基于条件生成对抗网络做像素预测任务,提出一个统一的图像到图像转译框架,自此敲开基于生成对抗网络的图像转译领域的大门。

计算机视觉中的图像转译是一个比较广泛的概念,即输入一幅图像,生成相对应的输出图像,并在转译过程中通过一定的约束使生成样本与真实样本在分布上尽量一致。基于生成对抗网络的图像转译有很多应用,比如,可以应用于图像去雨、图像着色等低级别任务,只改变图像表面特征的变化,而不会改变图像的内容;可以应用于季节变化、白天黑夜变化、梵高风格化和木炭风格化,在保证内容一致的情况下,对纹理做一定变化;可以应用于超分辨率重建,提高生成图像质量;可以应用于图像分割、卫星图像变地图,将真实图像映射为抽象图像,体现了图像转译在多到少映射上的应用;还可以应用于图像修复、文字变图像等。这些任务共有的特点是两个域相差不大,但当不成对的两个域在形状上存在较大差异且存在多个目标时,极大地提高了图像转译的难度。基于生成对抗网络的图像转译因其通用性,有广阔的发展空间,不仅可应用于空气中的图像,也可以应用于水下图像,我们输入一张水下图像,并将其转译至清晰的图像,可看作水下图像清晰化,水下图像清晰化任务将图像转译从空气中的图像应用至水下图像,进一步扩大了图像转译的应用范围。

\section{国内外研究现状}

\subsection{生成对抗网络的研究现状}
生成对抗网络(GAN)\cite{goodfellow2014generative}自2014年由Goodfellow等人提出后,引起学术界的关注,越来越多的研究者将其用于图像生成。生成对抗网络,顾名思义,是通过对抗的方式学习真实数据的分布,对抗需要至少两方相对,因此GAN主要由两部分组成:生成器和判别器。在训练的时候,生成器的目的是尽可能生成逼真的样本,判别器则尽可能判别生成样本和真实样本,二者在训练中使生成样本的分布逐渐接近真实样本的分布。目前对GAN的研究可分为两个方向,一个是研究GAN模型训练等理论问题,试图解决训练不稳定和模式崩塌等问题,二是研究GAN的应用,如GAN在图像生成、自然语言处理(NLP)或其它领域的应用。

因为训练方式是对抗的,所以容易出现训练时模型崩溃的情况,需要小心协调生成器和判别器的训练程度。DCGAN\cite{radford2015unsupervised}将卷积神经网络(CNN)引入GAN中,为CNN的网络拓扑结构加一系列约束使之可稳定训练,同时利用CNN更强的拟合和表达能力,生成更高质量的图像。DCGAN对GAN的改进集中在网络结构,而WGAN\cite{arjovsky2017wasserstein}则主要集中于损失函数的改进,WGAN认为交叉熵(JS散度)不适合衡量具有不相交部分的分布之间的距离,提出用wassertein距离衡量生成数据分布与真实数据分布之间的距离,理论上解决了训练不稳定的问题。WGAN-GP\cite{gulrajani2017improved}是在WGAN的基础上加入梯度惩罚,改进了连续性限制的条件,避免梯度消失和梯度爆炸,拥有比WGAN更快的收敛速度和更高的生成质量,且无需过多关注调参问题,增强训练稳定性。最小二乘生成对抗网络(LSGAN\cite{mao2017least})用最小二乘损失函数代替GAN中的损失函数,作者认为JS散度并不能使生成分布逼近真实分布,而最小二乘可以使生成分布尽可能逼近决策边界,模型在训练时更关注真实度不高的样本,解决了GAN训练不稳定以及生成图像质量差、多样性不足的问题。

除了GAN训练不稳定和模式崩塌的问题,模型压缩、预训练模型的利用和泛化等问题也纳入研究行列中。Li等人\cite{li2020gan}提出一个通用的压缩框架以减少条件生成对抗网络(CGAN\cite{mirza2014conditional})中生成器的推理时间和模型大小,在不降低图像生成质量的前提下,可减少现有部分网络的计算量,为交互式图像合成铺平了道路。尽管GAN在图像生成中取得巨大成功,但对已训练好的模型的重利用仍存在挑战。Gu等人\cite{gu2020image}提出一种新的逆映射方法,将已训练好的GAN模型作为有效的先验去处理图像着色、图像修复等多种任务。Bau等人\cite{bau2020rewriting}研究模型重写,提出一种公式,通过操纵深层网络的一层作为线性联想存储器来改变期望,旨在添加、删除和更改预先训练的深层网络的语义和物理规则。

GAN可以与迁移学习、半监督学习、多模态学习等结合,应用于图像处理和计算机视觉领域。Ledig等人\cite{ledig2017photo}将GAN引入超分辨率任务,提出的模型可在提升4倍分辨率的情况下恢复细小的纹理细节,增加逼真度。Pathak等人\cite{pathak2016context}结合编码器-解码器结构和GAN,提出一种基于上下文像素预测的无监督视觉特征学习算法,用于填补图像中缺失的区域,GAN在此学习图像特征并起到精细化生成图像修复区域的作用。Vondrick等人\cite{vondrick2016generating}基于GAN提出一个视频生成模型,从大量没有标签的视频中获取动态场景的先验信息,生成一些短小但质量高的视频,模型学习到的特征还可用于图像分类。Wu等人\cite{wu2016learning}提出三维生成对抗网络,建立了一个从低维概率空间到三维对象空间的映射,可生成三维物体。除以上所列,图像转译也是GAN的应用热点,GAN在此领域几乎无所不能。

\subsection{图像转译的研究现状}

% 图像转译根据是否需要成对数据分为有监督图像转译和无监督图像转译,
2017年,Isola等人基于条件生成对抗网络提出一个通用的图像转译框架pix2pix\cite{isola2017image},能够实现语义分割图转街景图、边缘图转真实图、卫星图转地图、图像着色等任务。针对成对图像转译任务中生成图像质量不高、局部伪影的问题,Wang等人提出的DRPAN\cite{wang2019discriminative}采用判别区域对抗学习的方式,设计了一个修正器用于改善图像转译的合成效果,可生成高质量的高分辨率图像。Park等人提出的SPADE\cite{park2019semantic}在给定语义分割图的情况下利用所设计的空间自适应归一化合成逼真的图像。
% Zhu等人提出的SEAN\cite{zhu2020sean}利用分割掩码图像控制生成图像的合成,提出一个语义区域自适应归一化,将区域风格编码和分割掩码作为输入,实现了对

上述三个工作需要大量的成对数据训练,是有监督图像转译,但大多数任务并没有成对数据集可用,对此,Zhu等人提出的CycleGAN\cite{zhu2017unpaired}、Yi等人提出的DualGAN\cite{yi2017dualgan}和Kim等人提出的DiscoGAN\cite{kim2017learning}都采用循环一致性损失训练网络,保证生成样本与输入样本是合理对应的。Liu等人提出的UNIT\cite{liu2017unsupervised}除了循环一致性思想,还提出一个共享潜在空间假设,假定来自两个域的图像可以被映射到共享潜在空间中相同的潜码,通过分解表示的方式实现无监督图像转译。

在图像转译的过程中,我们希望一张图像不局限于转译到一张图像,而是有多样的输出。Zhu等人将变分自编码器(VAE\cite{kingma2013auto})和生成对抗网络(GAN\cite{goodfellow2014generative})结合,提出的BicycleGAN\cite{zhu2017toward}可在成对数据集上解决图像多样性问题,作者提出在输出和潜在空间上加双向映射,不仅可以由潜在编码生成图像,也可以由图像得到对应的潜在编码,这可以防止两个潜在编码生成相同的输出,避免输出的单一性。在无监督图像转译方面,Huang等人提出的MUNIT\cite{huang2018multimodal}将图像分解为域不变的内容编码和特定于域的风格编码,假设两个域的内容编码映射在同一隐空间,通过将从两个风格隐空间中随机采样的多个风格编码与内容编码结合,从而实现多模态图像转译。Lee等人提出的DRIT\cite{lee2018diverse}也将图像分解到共享的内容隐空间和特定于域的属性隐空间,为了更好的分离内容编码和属性编码,还提出权重共享和内容判别策略。Mao等人提出的
MSGAN\cite{mao2019mode}提出模式搜寻正则化项,最大化生成图像之间的距离与隐向量之间的距离的比值,从而解决模式崩塌问题,生成更多样的结果。Chang等人提出的DSMAP\cite{chang2020domain}则通过将域不变内容空间中的特征重映射为特定于域的内容特征,进一步分离内容空间,可以实现更彻底的风格迁移。

当两个域存在较大形状变化时,如马到长颈鹿的转译,上述模型均不能实现较好转译。Gokaslan等人\cite{gokaslan2018improving}认为在图像转译中处理形变问题需要使用来自整个图像的空间信息,保持全局形状和局部纹理的一致性,因此在基于补丁的判别器中加入空洞卷积,使判别器对每个像素的判别都由全局上下文决定,除此之外,提出多尺度结构相似性感知重建损失来。Wu等人提出的TransGaGa\cite{wu2019transgaga}通过分解的方式,将图像空间分解到外表隐空间和几何隐空间的笛卡尔积上,分别建立外表和几何的转译,为了使网络学习两个隐空间独立但互补的表示,提出了几何先验损失和条件VAE损失,TransGaGa在人脸和猫脸、马和长颈鹿等有较大形变的数据集上实现高质量的图像转译。Kim等人提出U-GAT-IT\cite{kim2019u},将注意力机制引入有较大形变的转译中,与传统的注意力机制不同,U-GAT-IT并没有计算全局的权重,而是采用全局和平均池化下的类激活图(CAM),通过CNN确定分类依据的位置,使模型更关注于能够区分两个域的区域,从而实现转译。随着单目标形变图像转译研究的深入,Mo等人\cite{mo2018instagan}不再局限于单目标的转译,转而研究多目标形变图像转译,如将一张图像上的多只绵羊转译为多只长颈鹿,通过利用实例分割图,在保证背景不变的情况下对多目标进行转译。

随着图像转译在低级视觉任务上的成功,研究者们尝试将图像转译应用于水下图像中。Islam等人提出的UGAN\cite{fabbri2018enhancing}利用CycleGAN在不成对数据集中训练,将未失真的水下图像与生成的水下图像配对,通过基于条件生成对抗网络的图像转译模型实现图像增强。MCycleGAN\cite{lu2019multi}首先利用暗通道先验获得浑浊水下图像的透射图,然后对浑浊的图像和生成的清晰图像的三个颜色通道施以不同尺寸的滑动窗口来计算两种图像之间的SSIM\cite{wang2004image}损失,从而将水下风格的图像转译至清晰风格的图像。Guo等人提出的DenseGAN\cite{guo2019underwater}引入一个多尺度密集块算法,该算法采用密集连接、残差学习和多尺度网络进行水下图像增强,使原始的水下图像转译为相对清晰的图像。MLFcGAN\cite{liu2019mlfcgan}先提取多尺度特征,然后用全局特征增强每个尺度的局部特征以做颜色校正,使图像从水下的蓝绿色转译至正常的颜色。

\section{课题来源}

国家自然科学基金面上项目“类别不平衡条件下海洋浮游生物图像精细识别及其原位应用研究”(批准号:61771440)、国家自然科学基金面上项目“海洋中小型浮游生物原位光学观测关键技术研究”(批准号:41776113)。

\section{论文组织结构安排}

本文主要关注基于生成对抗网络的图像转译问题,并将其应用到多目标和水下两个任务中,具体安排如下:

第一章为绪论部分,该部分主要介绍图像转译的研究背景、应用价值和意义,并对生成对抗网络和图像转译的国内外研究现状做了简要概述。

第二章主要讨论生成对抗网络的基本原理、基于此改进的算法以及生成对抗网络在图像转译方面的应用,介绍了有监督图像转译和无监督图像转译,重点介绍了基于循环一致和基于分解表示的图像转译算法。

第三章介绍了无监督的单目标形变和多目标形变图像转译,并设计了基于循环一致的多目标形变图像转译算法,通过将背景、前景形状和前景纹理分开处理,成功地实现形状的变化,并向转译后的前景添加纹理,最后将前景与背景结合,从而得到多目标形变结果。

第四章讨论了水下图像清晰化问题,先介绍了水下成像特性,针对水下图像存在的固有问题分析,提出了一个基于分解表示的模型,将图像分解为色彩隐空间和信息隐空间两部分,设计的网络在水下数据集上进行对比实验,并通过消融实验验证了提出模块的有效性。

第五章对本文所做的工作和贡献进行综合讨论和总结,分析所提出方法的不足之处,同时对该方向的未来研究进行展望。
