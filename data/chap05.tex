\chapter{总结与展望}

\section{总结}

本文主要对现有的基于生成对抗网络的图像翻译问题进行了深入的探讨,调研了非配对图像翻译中的循环一致性方式以及分解表示,确定二者在图像翻译问题中的应用,并尝试用其解决两个问题。对于多目标形变图像翻译问题,我们将复杂的问题分为几个小问题,并引入循环一致性思想。对于水下图像清晰化问题,我们将分解表示与水下光学特性结合考虑,提出一个基于分解表示的图像翻译模型。主要工作内容如下:

\begin{itemize}
	\item [1.]
	调研并分析了当下基于生成对抗网络的图像翻译任务的特点,并对当前最新的研究算法进行了评估,在此基础上,我们主要关注非配对图像翻译中的两个问题:一是当图像中存在多个目标且两个域之间存在较大的形状差异时,存在无法使每个目标的形状和纹理都实现翻译的问题,二是水下图像因不可避免的退化导致其存在对比度低、模糊、有雾状效果和色偏的问题。文中对每个问题都提出了一个适用的基于生成对抗网络的图像翻译模型。

	\item [2.]
	针对多目标形变图像翻译任务,我们将这一极具挑战性的问题根据形状、纹理与背景分解为三个小问题,并逐个解决,然后经一个细化网络得到最终的翻译结果。形状翻译引入非配对图像翻译中的循环一致性思想,将实例分割图从一个域翻译至另一个域,完成形状的变化,纹理翻译利用对抗损失、重建损失和颜色损失使实例分割图翻译至对应的带有纹理的前景图像,背景翻译则利用效果优异的图像修复算法,得到相对完整的背景图像,然后利用生成的实例分割图、前景图像与背景图像融合出一张完整的图像,并输入到细化网络中进行更精细的优化。

	\item [3.]
	针对水下图像清晰化任务,我们从水下图像的光学特性出发,因为水下图像的退化主要是吸收和散射共同作用的结果,所以我们基于分解表示提出了一种新的算法,将吸收导致的色偏与散射导致的雾状效果分开考虑,假设水下图像域和清晰图像域都可分解为色彩空间与信息空间,我们对清晰图像的信息特征做深层特征提取,并在翻译前使水下图像域的信息特征与清晰图像域的信息空间对齐,从而提升模型在水下图像清晰化任务上的性能。

\end{itemize}

\section{展望}

在多目标形变图像翻译任务中,虽然我们的方法可以在有较大形状差异的两个域中对多个目标实现图像翻译,但从实验结果中可以看出,我们得到的结果仍有很大的提升空间,在形状和纹理的翻译部分可以生成更细致化的结果,只有当每个部分都取得令人满意的结果,才能保证最终生成高质量的结果。另外,我们现在利用实例分割图以确定图像中目标的数量、位置和形状等信息,后续可以在目前实验取得优异结果后思考如何去掉实例分割图的限制,使多目标形变图像翻译任务不再受数据集的约束,扩大模型的适用范围。

对于水下图像清晰化任务,相比较经典的和最先进的方法,我们的方法都有一定的优势,但对于浓雾区域的清晰化仍需研究,或许可以结合空气中的去雾模型与水下成像原理,最有效的算法需要后续探讨与实验后确定,除此之外,我们还可以再提升生成图像的清晰度,探究何种方式可以在不降低去雾性能的前提下适当提高对比度。
