\begin{resume}

  \resumeitem{个人简历}

  1995年12月10日出生于山东省烟台市。

  2014年9月考入成都理工大学信息科学与技术学院通信工程专业,2018年6月本科毕业并获得工学学士学位。
  
  2018年8月考入中国海洋大学信息科学与工程学院电子与通信工程专业攻读硕士学位至今。

  \researchitem{发表的学术论文} % 发表的和录用的合在一起

  % 1. 已经刊载的学术论文(本人是第一作者,或者导师为第一作者本人是第二作者)
  \begin{publications}
    \item Chao Wang$^\#$, Wenjie Niu$^\#$, Yufeng Jiang$^\#$, Haiyong Zheng$^\ast$, Zhibin Yu$^\ast$, Zhaorui Gu, and Bing Zheng. Discriminative Region Proposal Adversarial Network for High-Quality Image-to-Image Translation. IJCV, 2019.
  \end{publications}
  % 2. 尚未刊载,但已经接到正式录用函的学术论文(本人为第一作者,或者
  %    导师为第一作者本人是第二作者)。

  % 3. 其他学术论文。可列出除上述两种情况以外的其他学术论文,但必须是
  %    已经刊载或者收到正式录用函的论文。
  \begin{publications} % [before=\publicationskip,after=\publicationskip]
    \item Zonghui Guo$^\#$, Liqiang Zhang$^\#$, Yufeng Jiang, Wenjie Niu, Zhaorui Gu$^\ast$, Haiyong Zheng, and Bing Zheng, Guoyu Wang. Few-shot Fish Image Generation and Classification. Global OCEANS 2020: Singapore - U.S. Gulf Coast, 2020.
  \end{publications}
  \begin{publications} 
    \item Zonghui Guo, Haiyong Zheng$^\ast$, Yufeng Jiang, Zhaorui Gu, and Bing Zheng. Intrinsic Image Harmonization. CVPR, 2021.
  \end{publications}

  \researchitem{在校期间参加的研究项目} % 有就写,没有就删除
  \begin{achievements}
    \item 国家自然科学基金面上项目:类别不平衡条件下海洋浮游生物图像精细识别及其原位应用研究(批准号:61771440)。
  \end{achievements}
    \begin{achievements}
    \item 国家自然科学基金面上项目:海洋中小型浮游生物原位光学观测关键技术研究(批准号:41776113)。
  \end{achievements}

\end{resume}
